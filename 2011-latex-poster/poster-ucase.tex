\pdfminorversion=5
\pdfobjcompresslevel=2
\documentclass[usepdftitle=false,xcolor={svgnames}]{beamer}
\mode<presentation>
{
  \usetheme{ucase}
}
\usepackage{times}
\usepackage[spanish]{babel}
\usepackage[utf8x]{inputenc}
\usepackage[T1]{fontenc}
\usepackage[orientation=portrait,scale=1.5]{beamerposter}
\usepackage{multirow}
\usepackage{textcomp}
\usepackage{numprint}
\usepackage{csquotes}
\usepackage{calc,fp}
\usepackage{tikz}
\usepackage{listings}

\PrerenderUnicode{áéíóúÁÉÍÓÚñÑçÇ}

\title{\huge Grupo UCASE de Ingeniería del Software \\[.5ex] \LARGE TIC-025}
\author{Antonio García Domínguez}
\institute{Universidad de Cádiz}
\date{\today}
\hypersetup{%
  pdftitle=Grupo UCASE de Ingeniería del Software (TIC-025),%
  pdfauthor=Antonio García Domínguez}

\newcommand*{\financiador}[1]{\emph{#1}}
\newcommand*{\herramienta}[1]{\textbf{\structure{#1}}}

% Comentar para dejar de justificar el texto
\renewcommand{\raggedright}{\leftskip=0pt \rightskip=0pt plus 0cm} 

\newenvironment{autowidthdesc}[1]{%
  \begin{list}{}{\renewcommand\makelabel[1]{\structure{##1}\hfil}%
      \settowidth\labelwidth{\makelabel{#1}}%
      \setlength\leftmargin{\labelwidth+\labelsep}}}%
  {\end{list}}

\usetikzlibrary{chains,fit,positioning,shadows,shapes.geometric,shapes.multipart}

\makeatletter
\newcommand{\umlact@manual}{false}
\pgfkeys{
  /umlact/.cd,
  node/.store in=\umlact@node,
  users/.store in=\umlact@users,
  time limit/.store in=\umlact@tl,
  manual/.store in=\umlact@manual,
  type/.is choice,
  type/graph/.code={\gdef\umlact@stereotype{gpc}},
  type/activity/.code={\gdef\umlact@stereotype{pc}},
}
\newenvironment{umlactivity}[1][]{
  \begin{tikzpicture}[
    graphicnode/.style={
      draw,fill=black,color=black},
    initial/.style={
      graphicnode,shape=circle,inner sep=.2cm},
    final/.style={
      graphicnode,shape=circle,ultra thick,inner sep=.2cm},
    decision/.style={
      graphicnode,inner sep=.2cm,fill=white},
    fork/.style={
      graphicnode,inner sep=0,minimum width=0.75em,minimum height=2em},
    activity/.style={
      graphicnode,inner sep=0.8em,fill=white,rounded corners,text centered,
      text depth=0.1em,
    },
    cflow/.style={->,color=gray,very thick},
    condition/.style={auto,midway,color=black},#1
    alink/.style={dashed,very thick}
  ]
  \newcommand<>{\final}[2][]{
    \node##3[final,fill=white,##1] (##2) {};
    \node##3[final,inner sep=.125cm] (i##2) at (##2.center) {};
  }
  \newcommand<>{\decision}[2][]{
    \node##3[decision,##1,rotate=45] (##2) {};
  }
  \newcommand<>{\initial}[2][]{
    \node##3[initial,##1] (##2) {};
  }
  \newcommand<>{\fork}[2][]{
    \node##3[fork,##1] (##2) {};
  }
  \let\join=\fork
  \newcommand<>{\activity}[3][]{
    \node##4[activity,##1] (##2) {##3};
  }
  \newcommand<>{\perfannotation}[2][]{
    \begingroup
    \pgfkeys{/umlact/.cd,##2}
    \node##3[draw,##1,fill=white,rectangle split,
      rectangle split parts=4,
      rectangle split draw splits=false,
      rectangle split part align={center,left}]
     (\umlact@node){
      «\umlact@stereotype»
      \nodepart{second}
      concurrentUsers = \umlact@users
      \nodepart{third}
      timeLimit = \umlact@tl
      \nodepart{fourth}
      manual = \umlact@manual
    };
    \endgroup
  }
}{\end{tikzpicture}}
\makeatother

%%% Local Variables:
%%% mode: latex
%%% TeX-master: "slides-agd"
%%% End:


\begin{document}

\begin{frame}{}

  \begin{columns}[T]

    \column{.01\textwidth}

    \column{.7\textwidth}
    \begin{headerblock}{Miembros}
      \begin{center}
        \begin{columns}
          \column{.64\textwidth}
          \includegraphics[width=\textwidth]{grupo_15_sept_2011}
          \tikz[overlay]{
            \tikzstyle{peoplemarker}=[draw,shape=circle,fill=com1,text=white,inner sep=.1em,minimum width=.6em,text depth=0,text centered,semitransparent]
            \foreach \x/\n/\y in {
              -.853\textwidth/1/.66\textwidth,
              -.69\textwidth/2/.66\textwidth,
              -.563\textwidth/3/.66\textwidth,
              -.394\textwidth/4/.66\textwidth,
              -.195\textwidth/5/.66\textwidth,
              -.825\textwidth/6/.48\textwidth,
              -.655\textwidth/7/.48\textwidth,
              -.5\textwidth/8/.48\textwidth,
              -.325\textwidth/9/.48\textwidth,
              -.145\textwidth/10/.48\textwidth
            }{
              \node[peoplemarker] at (\x,\y) {\scriptsize \n};
            }
          }

          \column{.36\textwidth}
          \normalsize

          \structure{Responsable}

          \vspace{.5em}

          Inmaculada Medina Bulo\textsuperscript{8}

          \vspace{.5em}

          \structure{Investigadores}

          \vspace{.5em}

          Antonia Estero Botaro\textsuperscript{9}

          Antonio García Domínguez\textsuperscript{4}

          Francisco Palomo Lozano\textsuperscript{5}

          Guadalupe Ortiz Bellot\textsuperscript{10}

          José Antonio Jiménez Millán\textsuperscript{2}

          Juan Boubeta Puig\textsuperscript{1}

          Juan José Domínguez Jiménez\textsuperscript{3}

          Lorena Gutiérrez Madroñal\textsuperscript{7}

          Mª del Carmen de Castro Cabrera\textsuperscript{6}
        \end{columns}
      \end{center}
    \end{headerblock}

    \column{.265\textwidth}

    \begin{headerblock}{Proyectos en ejecución}
      \small

      \begin{center}
        \structure{Propios}
      \end{center}
      \begin{description}
      \item[TIN2011-27242] \enquote{Extensión de una metodología
          dirigida por modelos para SOA 2.0: prueba y adaptación de
          servicios}. \financiador{MICINN}.

      \item[PR2011-004] \enquote{Verificación, validación y
          adaptación en arquitecturas orientadas a servicios
          aplicando una metodología dirigida por
          modelos}. \financiador{UCA}.
      \end{description}

      \begin{center}
        \structure{En colaboración}
      \end{center}
      \begin{description}
      \item[TIN2008-02985] \enquote{Desarrollo dirigido por modelos de
          procesos de negocio en factorías software: aplicaciones a la
          Web 2.0 y arquitecturas multicapa en
          J2EE}. \financiador{MICINN}.
      \end{description}
    \end{headerblock}
  \end{columns}

  % See http://tex.stackexchange.com/questions/6370/how-to-center-a-beamercolorbox
  \begin{center}
    ~
    \setbeamercolor*{block title}{fg=white,bg=pri2}
    \begin{beamercolorbox}[ht=4ex,wd=.975\textwidth,center,colsep=.75ex,rounded=true]{block title}%
      \usebeamerfont*{block title}\large Líneas de investigación
    \end{beamercolorbox}
    ~
  \end{center}

  \vspace{-1em}

    \begin{columns}[T]
      \begin{column}{.45\textwidth}
        \begin{block}{Mutación evolutiva}
          \begin{columns}
            \column{.475\textwidth} \small La prueba de mutaciones
            realiza cambios (\emph{mutaciones}) en un programa para
            ver si sus pruebas pueden detectarlos. Proponemos la
            \emph{mutación evolutiva}, en que se mejoran los casos de
            prueba de forma automatizada usando algoritmos genéticos
            para localizar mutantes no detectados y generar nuevos
            casos que los detecten. Nuestra herramienta
            \href{http://neptuno.uca.es/~gamera}{\herramienta{GAmera}}
            implementa los algoritmos genéticos, usando los operadores
            de mutación para WS-BPEL 2.0 de
            \href{https://neptuno.uca.es/redmine/projects/sources-fm/}{\herramienta{MuBPEL}}.

            \column{.5\textwidth}
            \begin{center}
              \scriptsize \def\svgwidth{\textwidth}
              \input{mutacion-evolutiva.pdf_tex}
            \end{center}
          \end{columns}
        \end{block}

      \begin{block}{Pruebas dirigidas por modelos}
        \small Para obtener mejor software, se deberían realizar
        pruebas incluso antes de tener todo el código, usando modelos
        en su lugar. Dentro de la metodología
        \href{https://neptuno.uca.es/redmine/projects/sodmt/}{\herramienta{SODM+T}},
        hemos definido varios algoritmos de inferencia de requisitos
        de rendimiento en modelos de composiciones de servicios
        web. Estos requisitos se usarán después para generar casos de
        prueba concretos.

        \begin{center}
          % -*- latex -*-

\begin{umlactivity}
  % Activity diagram nodes
  \node (ini) {};
  \activity[below=.5em of ini,text=pri2]{aeo}{Evaluar}
  \decision[right=.5em of aeo,anchor=north west,ultra thick,inner sep=.5em]{decaccept}
  \fork[below right=1em and 1em of decaccept]{forkaccepted}
  \activity[right=3.5em of forkaccepted,anchor=west,text=pri2]{acso}{Enviar}
  \activity[below right=1.75em and .5em of forkaccepted,anchor=west,text=pri2]{aci}{Facturar}
  \activity[right=1em of aci,anchor=west,text=pri2]{app}{Pagar}
  \join[right=11.5em of forkaccepted,anchor=west]{joinaccepted}
  \join[right=16em of aeo]{joinaccept}
  \activity[right=.5em of joinaccept,anchor=west,text=pri2]{aco}{Cerrar}
  \node[above=.5em of aco] (fin) {};

  % Computed paths %%%%%%%%%%%%%%%%%%%%%%%%%%%%%%%%%%%%%%%%%%
  \coordinate (end1) at ([xshift=-.5em,yshift=-.25em]fin.north);
  \coordinate (end2) at (fin.north);
  \coordinate (end3) at ([xshift=.5em,yshift=.25em]fin.north);
  \coordinate (start1) at ([xshift=-1em,yshift=.-.25em]ini.north);
  \coordinate (start2) at (ini.north);
  \coordinate (start3) at ([xshift=1em,yshift=.25em]ini.north);

  \draw[->,line width=4pt,blue]
    (start1) -- ([xshift=-.5em]aeo.north)
    ([yshift=.5em]aeo.east) -- ([yshift=.125em]decaccept.north)
    ([yshift=.125em]decaccept.east) -- ([yshift=.5em]joinaccept.west)
    ([yshift=.5em]joinaccept.east) -- ([yshift=.5em]aco.west)
    ([xshift=-.5em]aco.north) -- (end1);

  \draw[->,line width=4pt,red]
    (start2) -- (aeo) -- (decaccept)
    (decaccept.south east) -- ([yshift=.5em]forkaccepted.west)
    ([yshift=.5em]forkaccepted.east) -- ([yshift=.5em]acso.west)
    ([yshift=.5em]acso.east) -- ([yshift=.5em]joinaccepted.west)
    (joinaccepted) -- (joinaccept.west)
    (joinaccept) -- (aco) -- (end2);

  \draw[->,line width=4pt,color=DarkGreen]
    (start3) -- ([xshift=.5em]aeo.north)
    ([yshift=-.5em]aeo.east) -- ([yshift=-.125em]decaccept.west)
    ([yshift=-.1em]decaccept.south) -- ([yshift=-.5em]forkaccepted.west)
    (forkaccepted.south) -- (aci.west)
    (aci.east) -- (app.west)
    (app.east) -- (joinaccepted.south)
    ([yshift=-.5em]joinaccepted.east) -- ([yshift=-.5em]joinaccept.west)
    ([yshift=-.5em]joinaccept.east) -- ([yshift=-.5em]aco.west)
    ([xshift=.5em]aco.north) -- (end3);

  \begin{scope}[every node/.style={draw,inner sep=0,shape=circle}]
    \node[color=blue,fill=white] at (start1) {3};
    \node[color=red,fill=white] at (start2) {2};
    \node[color=DarkGreen,fill=white] at (start3) {1};
  \end{scope}

  % Inferred and manual restrictions %%%%%%%%%%%%%%%%%%%%%%%%

  \begin{scope}[every node/.style={%
      draw,shape=ellipse,inner sep=0,fill=com3,anchor=south west}]
    \node at (aeo.south west)  {0.4};
    \node at (aci.south west)  {0.2};
    \node at (app.south west)  {0.2};
    \node at (aco.south west)  {0.2};
    \node at (acso.south west) {0.4};
  \end{scope}

\end{umlactivity}
        \end{center}
      \end{block}

      \vspace{-.18em}

      \begin{block}{Verificación formal de software}
        \small Usamos ACL2, un dialecto de LISP diseñado para
        verificación. Es también una lógica computacional y un sistema
        de razonamiento automatizado que ayuda a demostrar propiedades
        de programas.

        \begin{center}
          \begin{tabular}{c}
            \lstinputlisting[language=Lisp,basicstyle=\small\ttfamily,keywordstyle=\normalfont\small]{mergesort.lisp}
          \end{tabular}
        \end{center}
      \end{block}
    \end{column}
    
    \begin{column}{.45\textwidth}
      \begin{block}{Adaptación de servicios al contexto}
        \small Actualmente las propuestas para adaptar los servicios
        web al contexto se centran en la adaptación del
        cliente. Nosotros proponemos no sobrecargar los clientes
        llevando a cabo la adaptación de forma no intrusiva en el
        servicio mediante un desarrollo dirigido por modelos y
        orientado a aspectos.

        \vspace{.5em}

        \begin{center}
          \scriptsize \def\svgwidth{.8\textwidth}
          \input{adaptacionServicios.pdf_tex}
        \end{center}
      \end{block}

      \vspace{-.15em}

      \begin{block}{Procesamiento de eventos complejos}
        \small Mediante unos patrones de eventos, se pueden inferir
        nuevos eventos con mayor contenido semántico.  Estos eventos
        complejos pueden tratarse en tiempo real, agilizando la toma
        de decisiones.

        \vspace{1em}

        \begin{center}
          \scriptsize \input{patronCEP.pdf_tex}
        \end{center}
      \end{block}

      \vspace{-.1em}

      \begin{block}{Generación dinámica de invariantes (con SPI\&FM)}
        \begin{columns}
          \column{.025\textwidth}

          \column{.375\textwidth} \small Colaboramos con miembros del
          grupo SPI\&FM (UCA) en generar propiedades de ciertos puntos
          de una composición WS-BPEL a partir de trazas de ejecución,
          mediante la herramienta
          \href{http://neptuno.uca.es/~takuan}{\herramienta{Takuan}}.

          \column{.6\textwidth}
          \begin{center}
            \scriptsize \def\svgwidth{\textwidth}
            \input{testcase-improvement-cycle-color.pdf_tex}
          \end{center}
        \end{columns}
      \end{block}
    \end{column}
  \end{columns}

\end{frame}

\end{document}

%%%%%%%%%%%%%%%%%%%%%%%%%%%%%%%%%%%%%%%%%%%%%%%%%%%%%%%%%%%%%%%%%%%%%%%%%%%%%%%%
%%% Local Variables:
%%% mode: latex
%%% TeX-PDF-mode: t
%%% End:
