\documentclass[xcolor=svgnames]{beamer}

\mode<presentation>
{
  \usetheme{Warsaw}
  \setbeamertemplate{navigation symbols}{}
  \setbeamercovered{dynamic}
}

\usepackage[spanish,es-noshorthands]{babel}
\usepackage[utf8x]{inputenc}
\usepackage[T1]{fontenc}
\usepackage{csquotes}
\usepackage{tikz}
\usepackage{fancyvrb}
\usepackage[htt]{hyphenat}

\PrerenderUnicode{áéíóúÁÉÍÓÚçÇ}

\title[Pósters científicos]{¿Cómo elaborar un buen póster científico?}
\author{Antonio García Domínguez}
\date{Cursos de doctorado \\ \small 18 de enero de 2012}
\institute{Universidad de Cádiz \\\vspace{2em} \includegraphics{cc-by-sa}}

\AtBeginSection[]
{
  \begin{frame}<beamer>{Contenidos}
    \tableofcontents[currentsection,hideothersubsections]
  \end{frame}
}

\AtBeginSubsection[]
{
  \begin{frame}<beamer>{Contenidos}
    \tableofcontents[currentsection,subsectionstyle=show/shaded/hide]
  \end{frame}
}

\usetikzlibrary{calc,positioning,shapes,shapes.geometric}

\begin{document}

\begin{frame}
  \titlepage
\end{frame}

\begin{frame}{Contenidos}
  \tableofcontents[hideallsubsections]

  Materiales en \url{http://github.com/bluezio/phd-posters-session} y
  el Campus Virtual de la UCA.
\end{frame}

\appendix

\begin{frame}{Fin de la presentación}
  \begin{center}
    {\Huge ¡Gracias por su atención!}

    \vspace{3em}

    {\Large
      \href{mailto:antonio.garciadominguez@uca.es}{antonio.garciadominguez@uca.es}}
  \end{center}
\end{frame}

\end{document}
